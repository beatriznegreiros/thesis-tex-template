\chapter{State-of-the-art}
%\chapter{Theoretische Grundlagen}
\label{ch:stateart}


% This chapter can be also called Literature Review


\section{Previous works}
\label{sec:prevworks}
	
% For referring to sections or chapter you rpesented before:
As explained in section~\ref{sec:notlogic}
	
	
\section{Types of Something}
\label{sec:typesome}

		
\subsection{somethingsomething}
\label{subsec:somesome}
		
As the Table~\ref{tb:typesomething} shows.
		
\subsection{somethingnothing}
\label{subsec:somenoth}
		
\begin{table}[]
	\centering
	\label{tb:typesomething} % A good practice for labeling your objects in a .tex environment is to use acronyms for different things. For tables, use "tb:"
	\caption{Captions of tables should be positioned on the top, while figure captions should be in the bottom}
	\begin{tabular}{ll}
		\hline
		\textbf{Thing} & \textbf{Use} \\
		\hline
		something & something \\
		something & something \\
		something & something \\
		\hline
	\end{tabular}
	\end{table}
		
\section{Something Statistics}
\label{sec:somestatistics}
	
As shown in Equation~\ref{eq:happiness}
\begin{equation}\label{eq:happiness}
	happiness=\frac{EmptyCup+coffee}{EmptyCups}
\end{equation}	
	
	
\section{Nothing Logic}
\label{sec:notlogic}
	
\subsection{The logic underlying something}
\label{subsec:logicunderlying}
    	
    	
    	
\subsection{Concepts and terminology}
\label{subsec:conceptsterm}	
		
\subsubsection*{Something set rules} 

Understanding the semantics of something

\section{Something or nothing?}
\label{sec:someornot}

\subsubsection*{Some non-sense?}

    	
    	
    	
    